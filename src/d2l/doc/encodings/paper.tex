\documentclass[a4paper]{article}
%\def\year{2020}\relax
%\usepackage{aaai20}  % DO NOT CHANGE THIS
\usepackage{times}  % DO NOT CHANGE THIS
\usepackage{helvet} % DO NOT CHANGE THIS
\usepackage{courier}  % DO NOT CHANGE THIS
\usepackage[hyphens]{url}  % DO NOT CHANGE THIS
\urlstyle{rm} % DO NOT CHANGE THIS
\def\UrlFont{\rm}  % DO NOT CHANGE THIS
\frenchspacing  % DO NOT CHANGE THIS
%\setlength{\pdfpagewidth}{8.5in}  % DO NOT CHANGE THIS
%\setlength{\pdfpageheight}{11in}  % DO NOT CHANGE THIS

\usepackage[margin=1in]{geometry}

% our packages
\usepackage{amsthm}
\usepackage{amsmath}
\usepackage{amssymb}
%\usepackage{bm}
\usepackage{booktabs}
%\usepackage[inline]{enumitem}
%\usepackage{mathtools}
%\usepackage{multirow}
%\usepackage[mode=buildnew,subpreambles=true]{standalone}
\usepackage{subcaption}
%\usepackage{todonotes}
\usepackage[table]{xcolor}  % TODO Comment out for final submission
%\usepackage{pgfplots}
%\usepackage{tikz}
\usepackage{natbib}
\usepackage{fancyvrb}
\usepackage{amsfonts}



% Some useful macros
%\newcommand{\inlinecite}[1]{\citeauthor{#1}~(\citeyear{#1})}
%\newcommand{\citealp}[1]{\citeauthor{#1}~\citeyear{#1}}

\newcommand{\smallpar}[1]{{\vspace{10pt}\noindent \bf #1.}}

\newcommand{\free}[1]{\ensuremath{\mathrm{free}(#1)}}
\newcommand{\vars}{\ensuremath{\mathrm{vars}}}
\newcommand{\pre}{\ensuremath{\mathrm{pre}}}
\newcommand{\add}{\ensuremath{\mathrm{add}}}
\newcommand{\del}{\ensuremath{\mathrm{del}}}
\newcommand{\effs}{\ensuremath{\mathrm{effs}}}

\newcommand{\tup}[1]{\ensuremath{\langle #1 \rangle}}
\newcommand{\tuple}[1]{\tup{#1}}  % Just an alias
\newcommand{\set}[1]{\ensuremath{\left\{#1 \right\}}}
\newcommand{\setst}[2]{\ensuremath{\left\{#1 \mid #2 \right\}}}

\newtheorem{definition}{Definition}
\newtheorem{definitionandtheorem}[definition]{Definition and Theorem}
\newtheorem{theorem}[definition]{Theorem}
\newtheorem{lemma}[definition]{Lemma}
\newtheorem{proposition}[definition]{Proposition}
\newtheorem{corollary}[definition]{Corollary}
\newtheorem{example}[definition]{Example}

\newcommand{\wip}[1]{{\color{red} #1}}  % From "work in progress" :-)
\newcommand{\gfm}[1]{\footnote{\color{red}{[Guillem] #1}}}

\newcommand{\numtasks}[1]{\small{(#1)}}

\newcommand{\badtx}{\ensuremath{\mathrm{D}}}

\setcounter{secnumdepth}{0} %May be changed to 1 or 2 if section numbers are desired.

%PDF Info Is REQUIRED.
% For /Author, add all authors within the parentheses, separated by commas. No accents or commands.
% For /Title, add Title in Mixed Case. No accents or commands. Retain the parentheses.
 \pdfinfo{
/Title ()
/Author ()
} %Leave this

\title{Representation Learning for Generalized Planning - SAT Encodings}
%Your title must be in mixed case, not sentence case.
% That means all verbs (including short verbs like be, is, using,and go),
% nouns, adverbs, adjectives should be capitalized, including both words in hyphenated terms, while
% articles, conjunctions, and prepositions are lower case unless they
% directly follow a colon or long dash
%\author{GFM}


\begin{document}

\maketitle

\section{Learning of Generalized Features \& Policy}

Encoding is parametrized by
\begin{itemize}
 \item Pool $F$ of description logic features $f$, each with given feature complexity $\mathcal{K}(f)$.
 \item Training set consisting of a sample of transitions from a number of instances of the same domain.
       At the moment we're assuming the sample is complete, and we have full information on whether each state
       in the sample is a goal state, an unsolvable state, or otherwise (see below for definitions). We also have access
       to the minimum distance to a goal $V^*(s)$ for each state $s$.
 \item A parameter $\delta$ which is a ``slack'' value to determine the maximum deviation from the optimal $V^*(s)$
 what we will allow in our policy. This will be made clearer in the encoding below.
\end{itemize}


\subsection{Terminology}
A state is called \emph{reachable} if there is a path to it from $s_0$, and
it is called \emph{solvable} if there is a path from it to a goal
state.
A state is \emph{alive} if it is solvable, reachable and not a
goal state~\cite{frances-et-al-ijcai2019}.

We use $T$ to denote the set of all transitions $(s, s')$ in the training sample such that $s$ is alive.

\subsection{Improvements}

\paragraph{Non-distinguishability of transitions as an equivalence relation.}
Any fixed, given pool of features $F$ implicitly defines an equivalence relation where to transitions are
equivalent iff they cannot be distinguished \emph{by any feature in $F$}.
If two transitions cannot be distinguished by any feature, then clearly either the policy computed by the SAT solver
considers all of them as ``good'', or as ``bad''.
We'll exploit this by using one single SAT variable to denote whether \emph{all transitions in a given equivalence
class} are good or bad. When exploiting this notion of equivalence (which is implemented as an optional feature of
the CNF generator), then every mention below to SAT variable $Good(s, s')$ needs to be read as $Good(s_{\star}, s_{\star}')$,
where $(s_{\star}, s_{\star}')$ is the \emph{representative} transition of the equivalence class to which $(s, s')$ belongs.

\paragraph{Dead transitions.}
We use \badtx{} to denote the set of transitions that lead from one alive state to an unsolvable state.
If using equivalence relations, \badtx{} also includes all those other transitions in whose equivalence class
there is some other dead transition (computed as a fixpoint).


\newpage

\subsection{Hyperparameters}
\begin{itemize}
 \item $\delta \in \mathbb{N}_+$
 \item $\rho  \in \mathbb{N}_+$
\end{itemize}


\subsection{Variables}

\subsubsection{Main Variables}
\begin{itemize}
 \item $Good(s, s')$ for $s$ alive, $s'$ solvable and $(s, s') \not\in \badtx$.

 \item
 Let $R = \setst{s}{0 < V^*(s) \leq \rho}$ be the set of states that are at distance at most $\rho$ from some goal.
 We have variables $V(s, d)$ for $s \in R$ and $d \in [0, D]$, where $D = \delta \cdot \max_{s \in R} V^*(s)$.
 The intended meaning is to have a potential function $V(s)$ that ensures acyclicity of good edges up to a certain
 distance to the goal. Note that by definition $R$ only contains alive states.

 \item $Select(f)$, for each feature $f$ in the feature pool.
\end{itemize}

%\subsubsection{Auxiliary Variables}
%\begin{itemize}
% \item We'll use $GV(s, s', d)$ to denote that $Good(s, s')$ and $V(s') \leq d$. This is enforced with the constraint
% $GV(s, s', d) \rightarrow Good(s, s') \land V(s') \leq d$ for $s, s'$ alive, $(s, s') \not\in \badtx$, $d \in [0, D)$.
% The other direction of the implication is not necessary.
%\end{itemize}


\subsection{Base Constraints}

\smallpar{C1 (Optional)}
Goals are distinguishable from non-goals.
\begin{align}
\bigvee_{f \in D1(s, s')} Select(f),&\;\; \text{for $s$ goal, $s$ not a goal}
\end{align}


\smallpar{C2}
The policy has to be complete with respect to all alive states:
\begin{align}
\bigvee_{s' \text{ s.t. } (s, s') \in T \setminus \badtx} Good(s, s'),&\;\; \text{for $s$ alive.}
\end{align}

\smallpar{C3-4}
Good transitions can be distinguished from bad transitions.
Let $(s, s')$ and $(t, t')$ be \emph{representative} transitions
of two different equivalence classes such that $(s, s') \not\in \badtx$
(which implies that $s'$ is solvable). Then,

\begin{align}
 Good(s, s') \rightarrow Good(t, t') \lor
 Dist(s, s', t, t'),&\;\; \text{for $s, t$ alive, $(t, t') \not\in \badtx$.} \\
 Good(s, s') \rightarrow
 Dist(s, s', t, t'),&\;\; \text{for $s, t$ alive, $(t, t') \in \badtx$.}
\end{align}

\noindent where $Dist(s, s', t, t')$ is shorthand for $\bigvee_{f \in D1\&2(s, s', t, t')} Select(f)$.

\subsection{Acyclicity Constraints}

\subsubsection{Option 1: Goal Distance}


\smallpar{C5}
Variables $V(s, d)$ define a function that is total over the set of alive states,
and such that $V(s)$ is within lower bound $V^*(s)$ and upper bound $\delta \cdot V^*(s)$:
\begin{align}
 \bigvee_{V^*(s) \leq d \leq \delta \cdot V^*(s)} V(s,d),&\;\; \text{for $s \in R$.} \\
 \neg V(s, d) \lor \neg V(s, d')&\;\; \text{for $s$ alive, $1 \leq d < d' \leq D$.} \tag{\theequation${}^\prime$}
\end{align}


\smallpar{C6}
$V$ is always descending along Good transitions:
\begin{align}
 Good(s, s') \land V(s', d) \rightarrow \bigvee_{d < k \leq D} V(s, k),&\;\; \text{for $s, s'$ alive, $(s, s') \not\in \badtx$, $d \in [1, D)$.} \\
 V(s', D) \rightarrow \neg Good(s, s'),&\;\; \text{for $s, s' \in R$.} \tag{\theequation${}^\prime$}
\end{align}


%\smallpar{C4-5} Any upper bound on $V(s)$ (for $s$ not a goal) needs to be justified:
%\begin{align}
% V(s) \leq d+1 \rightarrow \bigvee_{\substack{
% s' \text{ goal child of } s\\
% (s, s') \not\in \badtx}} Good(s, s') \lor
% \bigvee_{\substack{
% s' \text{ alive child of } s\\
% (s, s') \not\in \badtx}} GV(s, s', d),&
% \;\; \text{for $s$ alive, $d \in [0, D)$.} \\
% \neg V(s) \leq 0,&\;\; \text{for $s$ not a goal.}
%\end{align}


\subsubsection{AC2. ASP}


\subsection{Soft Constraints}
We simply post a constraint $\neg Select(f)$ for each feature $f$ in the pool, with weight equal to its complexity $\mathcal{K}(f)$.

\section{Empirical Results}



\subsection{Blocks:on}
\begin{Verbatim}[fontsize=\footnotesize]
 1. clear(a) NILs AND clear(b) ADDs AND holding(·) ADDs AND n-clear NILs AND n-clear>0
 2. clear(a) ADDs AND clear(b) NILs AND clear(b)=0 AND holding(·) DELs AND n-clear INCs AND n-clear>0
 3. clear(a) ADDs AND clear(b) DELs AND holding(·) DELs AND n-clear NILs AND n-clear>0
 4. clear(a) DELs AND clear(b) NILs AND clear(b)>0 AND holding(·) ADDs AND n-clear DECs
 5. clear(a) NILs AND clear(a)=0 AND clear(b) DELs AND holding(·) ADDs AND n-clear NILs AND n-clear>0
 6. clear(a) ADDs AND clear(b) NILs AND holding(·) ADDs AND n-clear NILs AND n-clear>0
 7. clear(a) NILs AND clear(b) NILs AND holding(·) DELs AND n-clear INCs AND n-clear>0
 8. clear(a) ADDs AND clear(b) DELs AND holding(·) ADDs AND n-clear NILs AND n-clear>0
 9. clear(a) DELs AND clear(b) NILs AND holding(·) ADDs AND n-clear NILs AND n-clear>0
 10. clear(a) NILs AND clear(b) ADDs AND holding(·) DELs AND n-clear INCs AND n-clear>0
 11. clear(a) NILs AND clear(a)=0 AND clear(b) NILs AND holding(·) ADDs AND n-clear NILs AND n-clear>0
 12. clear(a) NILs AND clear(a)>0 AND clear(b) NILs AND clear(b)=0 AND holding(·) ADDs AND n-clear NILs AND n-clear>0
\end{Verbatim}

\subsection{Blocks:arbitrary}
(Atomic move encoding, experiment name: all$\_$at$\_$5)

\begin{itemize}
 \item 1 instance 5 blocks
 \item 11 iterations
 \item CNF Theory: 21795 vars + 189561 clauses
 \item 12659 features
 \item concept size bound=10,
 \item Number of transitions/equivalence classes: 11053/133
 \item Optimal MAXSAT solution with cost 9 found
\end{itemize}


\begin{Verbatim}[fontsize=\footnotesize]
1. n-clear INCs AND n-clear>0 AND n-ontarget DECs AND n-superficially-well-placed NILs AND n-superficially-well-placed>0
2. n-clear NILs AND n-clear>0 AND n-ontarget INCs AND n-ontarget>0 AND n-superficially-well-placed INCs AND n-superficially-well-placed>0
3. n-clear INCs AND n-clear>0 AND n-ontarget NILs AND n-ontarget>0 AND n-superficially-well-placed NILs AND n-superficially-well-placed>0
4. n-clear NILs AND n-clear>0 AND n-ontarget NILs AND n-ontarget>0 AND n-superficially-well-placed INCs AND n-superficially-well-placed>0
5. n-clear INCs AND n-clear>0 AND n-ontarget INCs AND n-ontarget>0 AND n-superficially-well-placed INCs AND n-superficially-well-placed>0
6. n-clear DECs AND n-ontarget INCs AND n-ontarget>0 AND n-superficially-well-placed INCs AND n-superficially-well-placed>0
7. n-clear INCs AND n-clear>0 AND n-ontarget NILs AND n-ontarget>0 AND n-superficially-well-placed DECs
\end{Verbatim}



\subsection{Spanner}

\begin{itemize}
 \item 8 instances with different spanner configurations, goal is always the same.
 \item 1 iteration
 \item CNF Theory: 466575 vars + 1732026 clauses
 \item 724 features
 \item concept size bound=8,
 \item Number of transitions/equivalence classes: 11053/133
 \item Optimal MAXSAT solution with cost 9 found
\end{itemize}

NOTE: Features of the form $f1>f2$ are not necessary for this solution.

\begin{Verbatim}[fontsize=\footnotesize]
0. bobs-loc-empty NILs AND bobs-loc-empty=0 AND n-items-somewhere DECs AND n-tightened NILs AND n-tightened=0
1. bobs-loc-empty NILs AND bobs-loc-empty>0 AND n-items-somewhere NILs AND n-items-somewhere>0 AND n-tightened NILs AND n-tightened=0
2. bobs-loc-empty DELs AND n-items-somewhere NILs AND n-items-somewhere>0 AND n-tightened NILs AND n-tightened=0
3. bobs-loc-empty NILs AND bobs-loc-empty=0 AND n-items-somewhere NILs AND n-items-somewhere>0 AND n-tightened INCs
4. bobs-loc-empty ADDs AND n-items-somewhere DECs AND n-tightened NILs AND n-tightened=0
\end{Verbatim}


\bibliographystyle{plain}
% Cross-referenced entries need to go after the entries that cross-reference them
\bibliography{abbrv-short,literatur,references,crossref-short}
\end{document}
